\documentclass[
  ngerman,
  paper=a4,
  10pt,
  headings=small,
  DIV=15,
]{scrartcl}

% use utf-8 encoding.
\usepackage[utf8]{inputenc}

% setup main font.
\usepackage{fontspec}
\setsansfont{Helvetica Neue}[
  BoldFont = {Helvetica Neue Bold},
]
\renewcommand{\familydefault}{\sfdefault}
\usepackage{microtype}

% space to leave in between answers.
\renewcommand{\arraystretch}{1.2}

% question environment (uses a table).
\newenvironment{question}[2]{
  \noindent
  \setcounter{answer}{1}
  \begin{tabular}{@{}cp{6.7cm}@{}}
  \textbf{#1} & \textbf{#2}\\
  \ifcsname #1\endcsname
    Hi & oops\\
  \fi
}{
  \end{tabular}
  \bigskip
}

\newcounter{answer}
\newcommand{\answer}[1]{
  \textbf{\Alph{answer}}
  \stepcounter{answer}
  & #1\\}
  
\usepackage{tikz}
\usetikzlibrary{circuits.ee.IEC}
\usetikzlibrary{positioning}
\newcommand{\graphic}[1]{&#1\\}
\tikzset{
  every circuit symbol/.style={draw,very thick},
  every circuit wire/.style={draw,very thick},
  every circuit/.style={draw,very thick,huge circuit symbols}}


% quotes setup
\usepackage{babel}
\usepackage{csquotes}
\MakeOuterQuote{"}

\title{Prüfungsfragen im Prüfungsteil "Technische Kenntnisse" bei Prüfungen zum Erwerb von Amateurfunkzeugnissen der Klasse E}
\date{September 2006}

\begin{document}
  % title page
  \maketitle
    
  % first page
  \newpage
  \vspace*{\fill}
  \noindent
  Dieser Fragen- und Antwortenkatalog basiert auf §~4 Abs.~1 Amateurfunkgesetz (AFuG) in Verbindung mit §~4 der durch Artikel~1 Ziffer~2 der Ersten Verordnung zur Änderung der Amateurfunkverordnung vom 25.~August 2006 (BGBl.~I~S.~2070) geänderten Verordnung zum Gesetz über den Amateurfunk (AFuV) vom 15. Februar 2005 (BGBl.~I~S.~242) in der Form, wie sie am 1.~Februar 2007 in Kraft tritt. Aus dem Katalog ersichtliche Einzelheiten werden erst ab dem 1.~Februar 2007 bei Amateurfunkprüfungen umgesetzt bzw. angewendet. Dazu erfolgt vor dem 1.~Februar 2007 eine entsprechende Veröffentlichung im Amtsblatt der Bundesnetzagentur.

\bigskip\noindent
Dieser Fragen- und Antwortenkatalog unterliegt den Bestimmungen des §~5 des Urheberrechtsgesetzes (UrhG).
Er kann jederzeit erweitert und aktualisiert werden. Neuauflagen werden im Amtsblatt der Bundesnetzagentur bekannt gegeben.
  \newpage
  \tableofcontents
  \twocolumn
  
  \include{technik_e_questions}
  \include{technik_e_formulas}
\end{document}




